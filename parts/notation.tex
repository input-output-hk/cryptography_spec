\section{Notation}
In this section we introduce some generic notation used throughout the spec. The
primitive-specific notation is introduced in the respective sections.

We denote by $\ec(\mathbb{F}_p)$ the finite abelian group based on an elliptic curve over a finite prime-order field
$\mathbb{F}_p$ (note that we simplify the notation and drop the explicit dependency on $\mathbb{F}_p$ and security
parameter $\kappa$). Most importantly, we assume the order of the group $\ec$ to be of the form $\cofvar\cdot \order$
for some small \emph{cofactor} $\cofvar$ (sometimes equal to 1) and large prime number $\order$, and that the (hence)
unique (sub)group $\group$ of order $\order$ is generated by a known base point $\generator$, i.e., $\group =
\langle\generator\rangle$,  in which the computational Diffie-Hellman (CDH) problem is believed to be hard. We use
$\hash$ to denote a cryptographically safe hash function, modeled as a random oracle, $\hash: \{0,
1\}^*\rightarrow\{0,1\}^{\ell(\kappa)}$.