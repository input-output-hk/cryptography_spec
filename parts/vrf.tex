\section{Verifiable Random Functions}
\label{sec:vrf}
\subsection{Summary}
\newcommand{\sk}{sk}
\newcommand{\vrf}{\texttt{VRF}\xspace}
\newcommand{\vrfsk}{\sk_{\vrf}}
\newcommand{\vrfvk}{\vk_{\vrf}}
\newcommand{\vrfoutput}{\beta}
\newcommand{\vrfproof}{\Pi}
\newcommand{\vrfkeygen}{\texttt{VrfKeyGen}}
\newcommand{\vrfgenerateproof}{\texttt{GenerateProof}}
\newcommand{\parallelsep}{\;||\;}
\newcommand{\true}{\texttt{true}}
\newcommand{\false}{\texttt{false}}


Verifiable Random Functions (\vrf) allow key-pair owners, $(\vrfsk, \vrfvk)$,
to evaluate a pseudorandom function in a provable way given a randomness seed.
Any party with access to the verification key, $\vrfvk$, the
randomness seed, the proof and the generated randomness can indeed verify
that the value is computed as expected. The VRF specification has changed
overtime in Cardano, and the nodes use a different algorith, pre-Babbage and
post-Babbage (if there exists a reference to the HF, we should point it out).
We expose both specifications, and the motivations for such a change.
\subsection{Generalised specification}
We use an additional hash function to that introduced in Section~\ref{sec:notation}. In particular
one that maps a binary input to an element of the group $\group$, $\hash_{s2c}: \{0,1\}^* \to \group$.

\subsubsection{Pre Babbage}
A \vrf function, in pre-Babbage eras, is defined by the following three algorithms:
\begin{itemize}
\item $\vrfkeygen(1^\secparam)$ follows the exact same procedure as $\keygen(1^\secparam)$ described in Ed25519, see Section~\ref{sec:ed25519}. Output key pair $(\secretkey, \vrfvk)$. We refer to the signing key $\signingkey$ in the Ed25519 section as $\vrfsk$.
\item $\vrfgenerateproof(\secretkey, \vrfvk, m)$ takes as input a keypair $(\secretkey, \vrfvk)$ and a message $m$, and returns the \vrf randomness $\vrfoutput$ together with a proof $\vrfproof$. Use $\secretkey$ to derive $\vrfsk$. Let $H \gets \hash_{s2c}(\vrfvk, m)$. Let $\Gamma \gets \vrfsk\cdot H$. Compute $r$ as defined in procedure $\sign$ from Section~\ref{sec:ed25519}. Let $c \gets \hash(H\parallelsep\Gamma\parallelsep k\cdot\generator\parallelsep k\cdot H)[..128]$. Compute $s \gets (r + c\cdot\vrfsk)\mod\order$. Finally, return the proof $\vrfproof \gets (\Gamma, c, s)$ and the randomness $\vrfoutput \gets \hash(\texttt{suite\_string}\parallelsep 0x03\parallelsep\cofvar\cdot\Gamma\parallelsep 0x00)$.
\item $\verify(m, \vrfvk, \vrfproof)$ takes as input a message $m$, a verification key $\vrfvk$ and a vrf proof $\vrfproof$, and returns $\vrfoutput$ or \false. It parses the proof as $(\Gamma, c, s) = \vrfproof$, and computes $H\gets\hash_{s2c}(\vrfvk, m)$. Let $U \gets s\cdot\generator - c\cdot\vrfvk$ and $V \gets s\cdot H - c\cdot\Gamma$. Compute the challenge $c'\gets\hash(H\parallelsep\Gamma\parallelsep U\parallelsep V)[..128]$. If $c'=c$, then return $\vrfoutput \gets  \hash(\texttt{suite\_string}\parallelsep 0x03\parallelsep\cofvar\cdot\Gamma\parallelsep 0x00)$, otherwise, return \false.
\end{itemize}
\subsubsection{Post Babbage}
A \vrf function, in post-Babbage eras, is defined by the following three algorithms:
\begin{itemize}
\item $\vrfkeygen(1^\secparam)$ follows the exact same procedure as $\keygen(1^\secparam)$ described in Ed25519, see Section~\ref{sec:ed25519}. Output key pair $(\secretkey, \vrfvk)$. We refer to the signing key $\signingkey$ in the Ed25519 section as $\vrfsk$.
\item $\vrfgenerateproof(\secretkey, \vrfvk, m)$ takes as input a keypair $(\secretkey, \vrfvk)$ and a message $m$, and returns the \vrf randomness $\vrfoutput$ together with a proof $\vrfproof$. Use $\secretkey$ to derive $\vrfsk$. Let $H \gets \hash_{s2c}(\vrfvk, m)$. Let $\Gamma \gets \vrfsk\cdot H$. Compute $r$ as defined in procedure $\sign$ from Section~\ref{sec:ed25519}. Let $c \gets \hash(H\parallelsep\Gamma\parallelsep k\cdot\generator\parallelsep k\cdot H)$. Compute $s \gets (r + c\cdot\vrfsk)\mod\order$. Finally, return the proof $\vrfproof \gets (\Gamma, k\cdot\generator, k\cdot H, s)$ and the randomness $\vrfoutput \gets  \hash(\texttt{suite\_string}\parallelsep 0x03\parallelsep\cofvar\cdot\Gamma\parallelsep 0x00)$.
\item $\verify(m, \vrfvk, \vrfproof)$ takes as input a message $m$, a verification key $\vrfvk$ and a vrf proof $\vrfproof$, and returns $\vrfoutput$ or \false. It parses the proof as $(\Gamma, U, V, s) = \vrfproof$, and computes $H\gets\hash_{s2c}(\vrfvk, m)$. Next, compute $c \gets \hash(H\parallelsep\Gamma\parallelsep k\cdot\generator\parallelsep k\cdot H)[..128]$. Finally, if $U = s\cdot\generator - c\cdot\vrfvk$ and $V = s\cdot H - c\cdot\Gamma$, then return $\vrfoutput =  \hash(\texttt{suite\_string}\parallelsep 0x03\parallelsep\cofvar\cdot\Gamma\parallelsep 0x00)$, otherwise, return \false.
\end{itemize}

This change allows for batch verification of proofs, which achieve up to a times two improvement in verification time, in exchange of a larger proof.
\subsection{Parameters of instantiation}
Some of the concrete parameter instantiations also differ between pre-Babbage and post-Babbage eras. We begin by describing those which coincide, and follow with a separate description for the ones that differ.
\begin{description}
\item[Parameter $\ell(\kappa)$ and suite $\texttt{suite\_string}$:] We set $\ell(\kappa)=512$. We choose the suite \texttt{ECVRF\_EDWARDS25519\_SHA512\_ELL2}, as defined in the standard draft~\cite{vrfdraft10}. This sets the parameter $\texttt{suite\_string}$ as $0x04$ and the following parameters.
\item[Curve:] We define the curve, and by consequence the finite prime order field, security parameter, cofactor, prime order subgroup and generator, as described in~\cite{CHES:BDLSY11}. In particular, we use Edwards25519 which is birationally equivalent to Curve25519~\cite{PKC:Bernstein06}.
\item[Hash:] As a hashing algorithm we use SHA512~\cite{FIPS1802}.
\end{description}

\subsubsection{Pre Babbage}
We proceed with the specifications on pre-Babbage eras.
\begin{description}
\item[Draft version:] pre-Babbage eras are build on top of Version 03 of the standards draft~\cite{vrfdraft03}.
\item[Deserialization:] A \vrf proof is represented by 80 bytes: the first 32 bytes, $b_{[..32]}$, represent the point $\Gamma$, the following 16 bytes, $b_{[32..48]}$, represent the scalar $c$, and the final 32 bytes, $b_{[48..]}$, represent the scalar $s$. A public key is also represented as 32 bytes, $b_{pk}$. Deserialization is valid only if:
\begin{itemize}
\item $b_{[..32]}$ when read as a little-endian integer, it is smaller than $p$.
\item $b_{pk}$ does not represent a low order point (by checking against a precomputed blacklist of size $\cofvar$, and when read as a little-endian integer, it is smaller than $p$.
\end{itemize}
\item[Hash to curve $\hash_{s2c}$:] \sloppy Elligator mapping, over a scalar computed by
hashing $\texttt{suite\_string}\parallelsep 0x01\parallelsep \vrfvk \parallelsep m$. The 
mechanism is described in \href{https://datatracker.ietf.org/doc/html/draft-irtf-cfrg-vrf-03#section-5.4.1.2}{Section~5.4.1.2} of version 3 of the draft~\cite{vrfdraft03}. Note that for 
implementing the mechanism as described in the draft, the sign bit is cleared before calling
the elligator function, meaning that the latter always works with sign = 0 (see \href{https://github.com/input-output-hk/libsodium/blob/tdammers/rebased-vrf/src/libsodium/crypto_vrf/ietfdraft03/convert.c#L84}{\texttt{\_vrf\_ietfdraft03\_hash\_to\_curve\_elligator2\_25519}}
function). See \href{https://datatracker.ietf.org/doc/html/draft-irtf-cfrg-vrf-03#appendix-A.4}{Appendix A.4} of version 3 of the draft for test vectors. 
\end{description}

\subsubsection{Post Babbage}
We finalize with the specifications of post-Babbage eras.
\begin{description}
\item[Draft version:] post-Babbage eras are build on top of a batch-compatible version of  Version 10 of the standards draft~\cite{vrfdraft10}. The specific construction is described in 
the technical spec studying the security of batch compatibility~\cite{batchspec}.
\item[Deserialization:] A \vrf proof is represented by 128 bytes: the first 32 bytes, $b_{[..32]}$, represent the point $\Gamma$, the following 32 bytes, $b_{[32..64]}$, represent the point $U$, the following 32 bytes, $b_{[64..96]}$, represent the point $V$, and the final 32 bytes, $b_{[96..]}$, represent the scalar $s$. A public key is also represented as 32 bytes, $b_{pk}$. Deserialization is valid only if:
\begin{itemize}
\item $b_{[..32]}$ when read as a little-endian integer, it is smaller than $p$.
\item $b_{[96..]}$ when read as a little-endian integer, is smaller than $\order$.
\item $b_{pk}$ does not represent a low order point (by checking against a precomputed blacklist of size $\cofvar$, and when read as a little-endian integer, it is smaller than $p$.
\end{itemize}
\item[Hash to curve $\hash_{s2c}$:] Hash to curve algorithm 
as defined in the hash to curve standard~\cite{h2cdraft13}, version 12, \href{https://datatracker.ietf.org/doc/html/draft-irtf-cfrg-hash-to-curve-12#section-6.8.2}{Section 6.8.2} (non uniform version). For test vectors, one can use those presented in \href{https://datatracker.ietf.org/doc/html/draft-irtf-cfrg-hash-to-curve-13#appendix-J.5.2}{Appendix J.5.2} of that same document. Reference implementation as called in the \href{https://github.com/input-output-hk/libsodium/blob/iquerejeta/ECVRF-EDWARDS25519-SHA512-TAI/src/libsodium/crypto_vrf/ietfdraft10/convert.c#L88}{libsodium fork}.
\end{description}
\subsection{External links}
We currently rely in the implementation available in the \href{https://github.com/input-output-hk/libsodium/tree/iquerejeta/ECVRF-EDWARDS25519-SHA512-TAI/src/libsodium/crypto_vrf}{fork of libsodium}. For Jormungandr we use a \href{https://github.com/input-output-hk/chain-libs/tree/master/chain-crypto/src/algorithms/vrf}{Rust implementation}. However, note that this implementation is not compatible with that of mainnet. We are actively working in a \href{https://github.com/input-output-hk/vrf}{compatible version} in Rust. However, we rely on the merge of a \href{https://github.com/dalek-cryptography/curve25519-dalek/pull/377}{PR} which has been inactive for a while. 
