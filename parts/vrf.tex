\section{Verifiable Random Functions}
\subsection{Summary}
\newcommand{\sk}{sk}
\newcommand{\vrf}{\texttt{VRF}}
\newcommand{\vrfsk}{\sk_{\vrf}}
\newcommand{\vrfvk}{\vk_{\vrf}}


Verifiable Random Functions (VRF) allow key-pair owners, $(\vrfsk, \vrfvk)$, 
to evaluate a pseudorandom function in a provable way given a randomness seed. 
Any party with access to the verification key, $\vrfvk$, the
randomness seed, the proof and the generated randomness can indeed verify 
that the value is computed as expected. 
\subsection{Generalised specification}
We denote by $\ec(\mathbb{F}_p)$ the finite abelian group based on an elliptic curve over a finite prime-order field $\mathbb{F}_p$ (note that we simplify the notation and drop the explicit dependency on $\mathbb{F}_p$ and security parameter $\kappa$). Most importantly, we assume the order of the group $\ec$ to be of the form $\cofvar\cdot \order$ for some small \emph{cofactor} $\cofvar$ and large prime number $\order$, and that the (hence) unique subgroup $\group$ of order $\order$ is generated by a known base point $\generator$, i.e., $\group = \langle\generator\rangle$,  in which the computational Diffie-Hellman (CDH) problem is believed to be hard.

We use $\hash$ to denote a cryptographically safe hash function, modeled as a random oracle, $\hash: \{0,1\}^*\rightarrow\{0,1\}^{\ell(\kappa)}$. We use another hash function that outputs an element in $\group$, $\hash_{s2c}: \{0,1\}^* \to \group$. 



\subsection{Parameters of instantiation}
\textit{This section covers the details of the primitive. It should cover all that is required for an external observer to implement a compatible version without needing to look into the code. This should cover, the curve parameters, hashing algorithms, format, serialisation/deserialisation mechanisms, etc.}

\subsection{External links}
\textit{Links of interest, such as papers, gdocs, jira pages, etc.}